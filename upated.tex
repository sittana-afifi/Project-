\documentclass[10pt,a4paper]{article}
\usepackage[utf8]{inputenc}
\usepackage{amsmath}
\usepackage{amsfonts}
\usepackage{amssymb}
\usepackage{graphicx}
\newtheorem{thm}{Theorem}
\newtheorem{proof}{Proof}
\begin{document}
	\section{Application of ZKP}
	In order to give the reader a motivation to explore further on Zero-Knowledge Proofs, we present in this section some interesting applications:
	\begin{enumerate}
		\item
		\begin{enumerate}
			\textbf{Applications of Zero-Knowledge Range proof \cite{morais2019survey:19}}
			let us mention a special kind of ZKP, Zero-Knowledge Range Proofs (ZKRP). ZKRP allows us to verify that a number lies within a certain range. Note that ZKRP holds the same three properties as ZKP, which are completeness, soundness, and zero-knowledge.
			\begin{itemize}
				\item
				\item Over 18 ZKRP (zero-knowledge range proofs):
				We can use this application in services that need to validate their user’s age before he gets to access their server. The interesting feature in ZKRP is that a user can validate his age without revealing it.
				\item 
				Know Your Customer (KYC):
				Since ZKRP allows us to validate that a certain amount of private information is in a specific range, we can use this property to ensure compliance while maintaining the user’s privacy.
				\item 
				
				Mortgage risk assessment:
				This application is more important for financial institutions, with it we can prove that the salary of someone is more than some amount to get a mortgage approved.
				\item 
				
				E-voting:
				By using ZKP we can construct a secure, and trusted e-voting system to ensure mutual authentication between the election authority server and the voters.
					\item 
				
				Electronic auctions and procurement:
				Using the protocol introduced in  \cite{palmer2010protocol:18} alongside  ZKP we can present a protocol that allows participants to ensure that an auction has run correctly without revealing the bid values of other participants while increasing the robustness of the auction protocol.
			\end{itemize}
		\end{enumerate}
		\item
		\begin{enumerate}
			\textbf{Authentication systems}: ZKP help the user who wants to verify its identity (password for example) via some secret to a second party, but the second party should not learn anything about this secret.\cite{zk}
		\end{enumerate}	
		\item
		\begin{enumerate}
			\textbf{Ethical behavior}:  ZKP uses to oblige a user to prove that his behavior is correct according to the protocol, because ZKP saves the privacy of the user’s secret during the process of providing the proof.\cite{zknowledge:13}
		\end{enumerate}	
		\item
		\begin{enumerate}
			\textbf{Nuclear disarmament}: ZKP assists inspectors to confirm whether or not an object is indeed a nuclear weapon without recording, sharing, or revealing the internal workings which might be secret.\cite{zknowledge:13}
		\end{enumerate}	
		\item
		\begin{enumerate}
			\textbf{DLT (Distributed ledger technology) and blockchain}: by using ZKP we can perform a valid transaction with keeping the sender, the recipient, and all other transaction details remain hidden:
			\begin{itemize}
				\item
				\item
				Confidential Transactions:
				
				In a usual transaction in DLT, the amount of money spent is public for anyone in the network, by using CT, each party can use a commitment scheme to hide the amount they send or receive. Hence, no adversary can see this amount.
				\item
				Provisions:
				
				Bitcoin exchanges work like banks, by holding their customer's bitcoin securely on their behalf, but these exchanges should have a proof of solvency, to avoid big problems like customer losing their money permanently. 
				A proof of solvency shows that the exchanges have sufficient reserves to settle each customer's account.
				Provisions (a privacy-preserving proof of solvency) is when an exchange could have a proof of solvency without revealing its Bitcoin addresses; total funds or liabilities; or any information about its customers.
			\end{itemize}
		\end{enumerate}	
	\end{enumerate}	
\section{ZKP for NP problems}
\cite{goldreich2019proofs:21} shows that all language in NP have Zero-Knowlege proof system, in this section we explore  ZKP for the graph three-colorability problem (G3C) and ZKP for Fiat-Shamir Identification Protocol, with short explanation, the protocol, and the idea of the proof for each one of them. 
\subsection{Graph Three-Colorability Problem (G3C)
\cite{morais2019survey:19}}
G3C is NP-complete, therefore it can be used for any language in NP, the protocol rests on a ``computational assumption``.

first, let's give the idea of G3C problem :
The idea of this problem is, we want to color a graph $G$ with 3 different colors such that no two adjacent vertices have the same color.  Suppose $G=(E,V)$ with, $\mid E\mid=m$ and $\mid V\mid=n$
The following protocol should apply $m^2$ times using independent coin tosses:
\begin{enumerate}
	\item The prover chooses random permutation $\Pi$ for the proper coloring ($\phi$) to get new coloring and put these colors into $n$ locked boxes and send them without their keys to the verifier.
	\item The verifier chooses randomly an edge $e\in E$ and sends it to the prover.
	\item Let e=(u,v), if $e\in E$ then the prover sends the corresponding keys to the verifier, else prover do nothing.
	\item The verifier opens the boxes and sends accept iff they are different.
\end{enumerate}

\begin{thm}
	G3C protocol satisfies completeness, soundness and zero-knowledge.
\end{thm}
\begin{proof}
	Completeness. Since $G \in G3C$, then for any edge $e \in E$, therefore the verifier will see any corresponding boxes they don’t have the same color.
	So the verifier returns accept for all $m^2$ iteration.
	Soundness. If $G\notin G3C$ then at least there is one edge has the same color on its vertices, so in each round the verifier will reject with probability at least $1/m$.
	The protocol satisfies soundness $(1/m)^{m^2}$
	Zero-knowledge. We have to construct a simulator $S$ such that it produces a transcript which is computationally indistinguishable from that produced from the protocol between the prover and the verifier.
	Since we are dealing with any verifier  define a black box $V^*$ to get the edge $e$ from it to avoid any attempt from the verifier to get more information from the verifier.
	Since $\pi$ also selected randomly and independently at each round, so the transcript that produced from $S$ is indistinguishable from the one between the prover and the verifier.
\end{proof}
 
\bibliography{references}
\bibliographystyle{ieeetr} 
\end{document}